\documentclass[8pt,a4paper]{article}
\usepackage{graphicx}
\usepackage{minted}
\usepackage{amsmath}
\usepackage[margin=0.5cm, landscape]{geometry}
%\usepackage{roboto}
\usepackage[T1]{fontenc}
\usepackage[utf8]{inputenc}
\usepackage{multicol}
\usepackage{tabu}
\usepackage{gitinfo2}
\usepackage{booktabs}
\usepackage{enumitem}
\usepackage{scrextend}
\usepackage{amsmath}
\usepackage{fontspec}
\usepackage{unicode-math}

%\makeatletter
%\input{\encodingdefault RobotoSlab-LF.fd}
%\DeclareFontShape{\encodingdefault}{\familydefault}{l}{it}{
% <->ssub*Roboto-LF/l/it
%}{}

\changefontsizes{5pt}

\setmainfont{Roboto Light}
\setmathfont{STIX Two Math}
\setmathfont{STIX Two Math}[version=script,StylisticSet=1]

\setlist[enumerate]{leftmargin=3mm, noitemsep}
\setlist[itemize]{leftmargin=3mm, noitemsep}

\graphicspath{{figures/}}

\usemintedstyle[java]{monokai}
\usemintedstyle[shell-session]{monokai}
\newminted[code]{java}{autogobble,
breaklines, 
frame=single,
bgcolor=black,
style=monokai,
tabsize=2
}
\newcommand{\m}[1]{\mintinline[bgcolor=black]{java}{#1}}
\newcommand{\f}[1]{
      \begin{figure}[H]
        \center
      \includegraphics[width=0.5\columnwidth]{#1}
      \end{figure}
}
\AtBeginEnvironment{code}{
    \fontsize{5}{5}\selectfont}

\DeclareMathOperator{\di}{d\!}

% LstListings
\usepackage{xcolor}   % for \textcolor

\usepackage{titlesec}
\titlespacing*{\section}{0pt}{1pt}{0pt}
\titlespacing*{\subsection}{0pt}{1pt}{0pt}


\setlength{\columnseprule}{0.2pt}
\setlength{\columnsep}{10pt}
\renewcommand{\columnseprulecolor}{\color{lightgray}}


\begin{document}
\begin{multicols}{5}
  \noindent
  \textbf{Multimedia Processing - Riassunto} \\
  \gitAbbrevHash - \gitAuthorIsoDate \\

    \section{Suono}
    Forma di energia, sulla terra vibrazione = suono. \\
    \textbf{Velocità}\quad 0° C:\quad $340.27\ [m/s]$,
    \quad 20° C:\quad $343\ [m/s]$ (mach 1). \\
    Nell'acqua, il suono viaggia 4.3 volte più veloce che nell'aria.
    Nel vuoto non c'è suono. Oggetti a velocità superiore a mach 1 viaggiano a
    velocità supersonica. \\
    \subsection{Energia Sonora}
    Joule [J], il mezzo di propag. (aria, acqua, ...) funge da accumulatore
    di energia, sia potenziale che cinetica.
    \begin{equation*}
      W = W_{\text{potential}} + W_{\text{kinetic}} = \int_V \frac{p^2}{2\rho_0c^2} \di V
      + \int_V \frac{\rho v^2}{2} \di V
    \end{equation*}
    dove:\\
    \begin{tabular}{l l}
      $V$ & è il volume d'interesse\\
      $p$ & è la pressione sonora\\
      $v$ & è la velocità delle particelle\\
      $\rho_0$ & è la densità del mezzo in assenza di suono\\
      $\rho$ & è la densità locale del mezzo\\
      $c$ & è la velocità del suono
    \end{tabular}
    \subsection{Pressione Sonora}
    Detta anche pressione acustica, è rappr. dalla variazione di pressione esercitata
    dal suono rispetto alla pressione atmosferica ambientale.
    Unità di misura: Pascal [Pa]
    \begin{equation*}
      p_{\text{total}} = p_{\text{stat}} + p 
    \end{equation*}

    \begin{tabular}{ll}
      $p$ & è la pressione sonora \\
      $p_{\text{total}}$ & è la pressione totale \\
      $p_{\text{stat}}$ & è la pressione ambientale
    \end{tabular}

    \subsection{Pressione Sonora Effettiva (RMS)}
    Valore medio di pressione istantanea calcolata su un intervallo di tempo
    definito. La media RMS viene anche detta media quadratica.
    
    \subsection{Livello di pressione sonora (SPL)}
    Valore relativo calcolato come il logaritmo del rapporto fra una data pressione
    sonora e una pressione sonora di riferimento. Nell'aria si usa $20\ [\mu \text{Pa}]$,
    $1\ [\mu \text{Pa}]$ per altri mezzi. Si tratta del limite di capacità d'udito 
    dell'essere umano.

    \subsection{dB SPL}
    Il livello di pressione sonora si misura in dB SPL:
    \begin{equation*}
      L_p = 20 \log_{10} (\frac{p_\text{rms}}{p_\text{ref}})\ \text{dB}
    \end{equation*}
    Una pressione RMS di $1\ [\text{Pa}]$ corrisponde ad un livello di $94
    \ [\text{dB SPL}]$.
    \subsection{Potenza sonora ($L_w$)}
    Energia Acustica emessa da una sorgente. Valore assoluto. Indipendente dalla
    distanza e dalle dim. della stanza. È la potenza totale emessa dalla sorgente
    in tutte le direzioni. Misurato in dB SWL. Viene utilizzato p.es per misurare
    la quantità di rumore prodotto dagli utensili di lavoro.

    \subsection{Psicoacustica}
    Scienza che studia la percezione del suono e la risposta psicologica prodotta.
    Usata nella localizzazione delle sorgenti sonore, riconoscimento dei timbri e
    negli effetti di mascheramento. Possiamo ingannare l'orecchio approfittando degli
    effetti psicoacustici. E.g: stereofonia / compressione psicoacustica dell'MP3.

    \subsection{Suoni}
    Suoni periodici vs aperiodici.\\
    \textbf{Ampiezza} determina l'intensità del suono. \\
    \textbf{Lunghezza d'onda} e \textbf{frequenza}: per i suoni periodici determinano
    l'altezza (pitch) del suono.\\
    La frequenza è proporzionale alla lunghezza d'onda. Per una maggiore lunghezza
    d'onda si avrà una minore frequenza.

    \subsubsection{Periodo e Frequenza}
    La lunghezza d'onda di un suono ne determina il periodo: è il tempo
    necessario per completare un ciclo di vibrazione. Più il periodo è lungo, più la
    frequenza (e di conseguenza la nota) sarà bassa. \\
    Periodo $T$ misurato in secondi.
    \begin{equation*}
      F = \frac{1}{T}\ [\text{Hz}]
    \end{equation*}
    \noindent
    L'uomo sente da $20$ Hz a $20$ kHz, con percezione migliore fra i $3$ e $4$ kHz.

    \subsubsection{Fase}
    Varia a dipendenza dell'istante in cui l'onda comincia la propria oscillazione.
    Solitamente inaudibile, ha però un forte effetto quando vengono sommate.

    \subsubsection{Timbro}
    Anche chiamato colore del suono.
    Determina il carattere del suono. Strumenti diversi producono suoni di timbro
    diverso. Timbro più semplice: sintetizzatore di sinusoide, timbro più complesso:
    rumore bianco. \\
    La percezione del timbro è determinata da due caratteristiche fisiche: lo spettro 
    e l’inviluppo.

    \columnbreak

    \subsubsection{Spettro}
    Ogni suono può essere modellato come combinazione di onde sonore sinusoidali a
    differenti frequenze. Se un suono contiene freq. che sono multipli di una freq.
    fondamentale, l'unione di questi suoni viene percepita come un unico suono
    la cui frequenza è quella fondamentale e il cui carattere è det. dalle freq.
    e ampiezze coinvolte. \\
    Per un tono puro (una sinusoide), il grafico è formato da una semplice linea 
    verticale, per un suono musicale da una serie di linee in corrispondenza alle 
    freq. fondamentali ed alle armoniche. Per un rumore lo spettro è costituito da una
    banda di frequenza.
    
    \f{noise-types}

    \subsubsection{Inviluppo}
    Andamento dell'ampiezza nel tempo. Ogni timbro ha un suo particolare inviluppo. \\
    L’inviluppo di un suono è solitamente composto da 4 fasi, dal momento in cui viene 
    originato, al momento in cui si estingue: attack, decay, sustain, release.

    \subsubsection{Spettrogramma}
    Lo spettrogramma è la rappresentazione grafica dell'intensità di un suono in funzione
    sia del tempo, che della frequenza. È lo strumento migliore per osservare il timbro
    di un suono. Ne esistono due varianti: bidimensionale e tridimensionale.
 
    \subsubsection{Direzione e distanza dal suono}
    Il sistema uditivo degli esseri umani è capace di distinguere la direzione di 
    provenienza e di stimare la distanza della sorgente di un suono. Tendenzialmente 
    la direzione viene identificata soprattutto sfruttando la differenza tra l’istante 
    di arrivo del suono ad un orecchio, rispetto all’altro. Mentre per la stima della 
    distanza, vengono soprattutto sfruttati i riverberi (prodotti dall’ambiente), in
    particolare l’informazione di predelay (il tempo che intercorre fra l’arrivo del
    suono diretto e le prime riflessioni).
    \textbf{Head related transfer function (HRTF)}
    Quando il suono si scontra con l’ascoltatore, le caratteristiche fisiche dei 
    ricettori (forma della testa, orecchie, etc.) modificano il suono e quindi la 
    percezione ottenuta.
 
    \subsubsection{Stereofonia}
    L’ascolto stereofonico è un inganno psicoacustico in cui vengono utilizzate due
    sorgenti sonore (correttamente posizionate), con una tecnica di riproduzione in
    cui i medesimi contenuti sonori vengono riprodotti con differenti ampiezze.
    Quest’inganno permette di posizionare sorgenti sonore virtuali nello spazio
    compreso fra i due altoparlanti.


    \section{Audio Analogico vs Audio Digitale}
    \textbf{Analog to Digital Converter (ADC)} Un sistema che converte un segnale 
    analogico in un segnale digitale. \\
    \textbf{Digital to Analog Converter (DAC)} Un sistema che converte un segnale 
    digitale in un segnale analogico.

    \subsection{dBV e dBu}
    Unità di misura usate per intensità di voltaggio.
    Sono valori specifici a valori di riferimento. \\
    dbV: voltaggio RMS relativo a 1 [V]\\
    dbu: voltaggio RMS relativo a $\sqrt{0.6} V \approx 0.7746 V \approx -2.218 dBV$

    \subsection{Range dinamico}
    Rapporto fra ampiezza e onda sinusoidale più forte possibile (non distorta)
    e rumore di fondo (rumore RMS) del sistema in uso. \\
    Spesso confuso con SNR, che è la differenza fra livello medio del segnale
    e livello medio del rumore di fondo. Il range dinamico dell'orecchio umano
    è attorno ai 140 dB. Dai 120 dB a 140 c'è la soglia del dolore.

    \subsection{Livelli ed Headroom}
    Livelli di voltaggio negli standards sono detti livelli nominali.
    L'headroom è la differenza fra il livello nominale e il livello massimo,
    oltre il quale il segnale viene compromesso dalla distorsione.

    \subsection{VU meters}
    Usati per rappresentare livelli di intensità sonora in apparecchi come mixer.
    Unità di misura: Volume Unit (VU). In sit. normali, 0VU permette un headroom
    di almeno 18dB senza distorsione significativi. Solitamente negli apparecchi
    professionali 0 VU = +4 dBu, mentre in quelli consumer 0 VU = -10 dBV

    \subsection{Range dinamico di segnali digitali}
    Dettato dall'errore di quantizzazione. 
    \begin{tabular}{ccccccc}
      bits & 8 & 12 & 16 & 18 & 20 & 24 \\
      dB & 48 & 72 & 96 & 108 & 120 & 144 
    \end{tabular} \\
    Qualità CD: 96 dB. Esistono algoritmi di noise shaping (dithering) capaci di
    spostare la perc. del range din. a 120 dB.
    Per il processing si usano 32 bits, float.

    \subsection{Pulse Code Modulation (PMC)}
    La tecnica di conversione più utilizzata è quella di PCM: ad essere campionata 
    a intervalli regolari è l’ampiezza del segnale analogico. Ogni campione viene 
    quantizzato al valore più vicino all’interno del range dei possibili valori 
    digitali. \\
    Esistono quindi due caratteristiche principali di un segnale audio: bit depth e
    sample rate.

    \subsubsection{Bit Depth}
    Quantità di bits utilizzati, influenza la qualità sonora in termini di dinamica
    massima acquisibile.

    \subsubsection{Sample Rate}
    Frequenza di campionatura. Con 44.1 kHz è possibile campionare freq. fino ad un
    massimo di 22050 Hz.

    \subsubsection{Nyquist-Shannon}
    \begin{equation*}
      F_s = 2 \cdot F_{max}
    \end{equation*}
    

    \subsection{Decibels Full-Scale (dbFS)}
    Per misurare l’ampiezza dei segnali digitali viene utilizzata l’unità di misura
    dBFS. È un’unità di misura che mette in relazione l'ampiezza del segnale con 
    l’ampiezza del picco massimo permesso da quel sistema (detto il full-scale). Oltre 
    alla soglia del full-scale interviene il clipping, una forma di distorsione molto 
    aggressiva. Al full-scale è assegnato il valore di 0 dBFS, quindi tutti i valori 
    più piccoli del massimo sono negativi.

    \subsection{Aliasing}
    Capita quando si campionano frequenze oltre alla frequenza di Nyquist.
    Le freq. oltre alla freq. massima vengono riflesse all'interno dello spettro
    generando freq. inesistenti.

    \subsection{Total Harmonic Distortion (THD)}
    Valore RMS degli armonici prodotti dai convertitori relativi al livello RMS di
    un segnale di input sinusoidale vicino a full-scale.

    \subsection{Formati audio per files}
    \begin{itemize}
      \item Formati non compressi (Dati PCM/RAW, .wav, .aiff, .au) 
      \item Formati con compressione lossless (Simil-zip, .flac, .ape, .wv)
      \item Formati con compressione lossy (.mp3, .aac, .opus)
    \end{itemize}
    \noindent
    PCM, 16 bits, 44.1 kHz, stereo: $16 \cdot 44100 \cdot 2 = 1411200\ \text{bps}$\\
    MP3 256, comprime al massimo $256$ kbps

    \section{Tecniche base di audio processing}
    \subsection{Nozioni}
    \textbf{Risposta impulsiva} La risposta impulsiva di un sistema è una funzione 
    (del tempo) che coincide con l’output del sistema se sollecitato con un impulso. \\
    \textbf{Risposta in frequenza} La risposta in frequenza è una misura quantitativa 
    dello spettro di output di un sistema, utilizzata per caratterizzare la dinamica 
    del sistema. \\
    \textbf{Risposta in fase} In maniera simile alla risposta in frequenza è possibile 
    rappresentare la risposta in fase: mette in relazione le differenze di fase fra il 
    segnale di input e quello di output di un sistema.

    \subsection{Trasformata di Fourier}
    Scompone una funzione del tempo (quindi un segnale) nelle frequenze che la
    compongono. Risulta una funzione di numeri complessi, il cui valore assoluto
    rappresenta l'ampiezza di ogni frequenza del segnale, mentre l'argomento
    complesso è la differenza di fase dalla sinusoide di quella determinata frequenza.\\
    Permette quindi di passare dal dominio del tempo al dominio delle frequenze.
    La sua inversa si chiama sintesi di Fourier.

    \subsection{Filtri: Sistemi LTI (Linear Time Invariant System)}
    \textbf{Time-invariant} Sistema che dato un input in diversi istanti di tempo, 
    restituisce lo stesso output. \\ \\ 
    Nella realtà, molti sistemi possono però essere modellati per semplicità come 
    sistemi LTI, trascurando le eventuali non-linearità. I sistemi LTI sono importanti 
    perché possono essere risolti con metodologie di signal processing. E’ possibile 
    realizzare filtri con l’FFT, alternative più efficaci sono i filtri FIR e IIR.


    \subsubsection{Filtro FIR}
    Un filtro con finite impulse response (FIR) è un filtro la cui risposta
    impulsiva (o risposta a qualsiasi input di lunghezza finita) è di durata finita,
    perché l'output va automaticamente a 0 in un tempo finito.
    La risposta impulsiva di un FIR di ordine $N$ dura esattamente $N + 1$ campioni.

    \subsubsection{Filtri IIR}
    Filtri con risposta impulsiva infinita. A differenza di FIR hanno un feedback,
    per questo motivo la risposta in freq. dei filtri IIR è migliore di quella dei
    filtri FIR dello stesso ordine.
    IIR sono più efficienti e possono essere facilmente parametrizzati in tempo reale.
    La risposta in fase non è lineare. Se non vengono disegnati correttamente possono
    diventare instabili a causa dell'anello di feedback.

    \section{Nozioni base sull'immagine}
    \subsection{Immagine}
    Una rappresentazione della percezione visiva umana di un soggetto o di un ambiente.
    Può essere bidimensionale (foto) o tridimensionale (ologramma). Le immagini 
    catturano la luce e possono essere generate con dispositivi ottici o dagli umani 
    tramite l’occhio.

    \subsection{Luce}
    Radiazione elettromagnetica (EMR) in una certa porzione dello spettro 
    elettromagnetico. \\
    \textbf{EMR} Classificata per lunghezza d’onda in radio, micro onde, infrarosso, 
    luce visibile, ultravioletto, raggi X e raggi gamma . \\
    \textbf{Lunghezza d'onda} 400 - 700 nm (fra 430 e i 750 THz). \\
    Energia radiante [J]\\
    Flusso radiante [W]\\
    \textbf{Radianza} Watt allo steradiante per metro quadrato. \\
    \textbf{Luce visibile} Viene emessa e assorbita in piccoli pacchetti chiamati 
    fotoni, che esibiscono proprietà sia delle onde che delle particelle, questa 
    caratteristica è la dualità onda-particella. \\
    \textbf{Ottica} Studio della luce e della sua interazione con la materia 
    (fenomini ottici come arcobaleno ecc.) \\ 

    \subsubsection{Grandezze per misurare la luce}
    \textbf{Radiometriche} Relative alla radiazione elettromagnetica, 
    la misurazione è compito della radiometria. Sono: Energia radiante [J],
    Flusso radiante [W] e Radianza [W allo  steradiante per $m^2$]. \\
    \noindent
    \textbf{Fotometriche} Quantificano l’emissione luminosa partendo dalle 
    grandezze radiometriche ma mediante pesatura con la curva di risposta 
    spettrale dell’occhio umano. Sono: Flusso luminoso [Lumen], Intensità 
    luminosa [Canela], Illuminamento [Lux].

    \subsection{Colori}
    I colori vengono generati dagli oggetti in funzione della loro capacità
    di riflettere le varie lunghezze d’onda presenti nella luce visibile.
    L'occhio umano può distinguere fino a 10M di colori. \\
    \subsubsection{Modelli}
    \textbf{Adattiva} somma dei colori derivati dai primari $\rightarrow$ bianco 
    (somma colori modello RGB). \\ 
    \textbf{Sottrattiva} somma dei primari (partendo dai secondari) $\rightarrow$ nero 
    (somma colori modello CMY). \\
    \f{color-mode}
    \subsubsection{Parametri}
    \textbf{Hue} Gradazione o varietà di colore\\
    \textbf{Colorfulness, Saturation e Chroma (da non confondere con Video Chroma)}
    Rappresentano l'intensità cromatica\\
    \textbf{Brightness e Lightness} Rappresentano la percezione di intensità luminosa.\\
    \textbf{Contrast} rappresenta la differenza in intensità luminosa o colore dell'immagine.
    \subsubsection{Istogramma dei colori}
    Rappresentazione della distribuzione dei colori. Se l'immagine è monocromatica
    $\rightarrow $ istogramma di intensità.

    \subsection{Parallasse}
    Differenza della posizione apparente di un oggetto visto da due differenti punti di 
    vista.

    \section{Immagine digitale}
    \subsection{Vettoriale vs Raster}
    \textbf{Vettoriali} indipendenti dalla risoluzione, sono però più laboriose da 
    produrre. \\
    \textbf{Raster} Non possono essere adattate alle risoluzioni senza perdita di qualità.
    \subsection{Sensori CCD o CMOS}
    Charge-Coupled Devices (CCD) e Complementary Metal-Oxide-Semiconductor (CMOS).
    Un CCD ha un unico amplificatore per ogni pixel.
    La maggior parte delle camere digitali ha sensori CMOS
    perché offrono prestazioni migliori, sono più veloci e consumano meno. 
    I CCD vengono ancora utilizzati per fotocamere digitali a basso costo.

    \subsection{Campionatura e Quantizzazione}
    Strumenti come i frame grabber permettono di discretizzare le immagini.
    L'immagine deve essere discretizzata sia spazialmente (sampling)
    sia in termini di intensità di luce (quantization). \\
    I fattori che influenzano la digitalizzazione sono:
    \begin{itemize}
      \item Sampling rate: determina la risoluzione spaziale
      \item Livello di quantizzazione: determina la tonalità di grigio o livelli
        di colore che vengono acquisiti. 
    \end{itemize}

    \subsection{Risoluzione Spaziale}
    Misura del più piccolo dettaglio discernibile in un'immagine digitale.
    Per misurarla si usano i DPI (dots per inches). Affermare che un'immagine ha una
    risoluzione di MxN pixels non ha un effettivo significato se la risoluzione non è 
    messa in relazione ad un'unità di misura spaziale.

    \subsection{Risoluzione di intensità}
    Più piccolo dettaglio discernibile nel livello di intensità di grigio o colore.
    Per le immagini grayscale vengono d'abitudine usati 8 bits (256 valori), mentre
    per le immagini a colori si usano 16 bits. L'utilizzo di 32 bits è raro.

    \subsection{Colori nelle Schede Video}
    Le schede video attuali dedicano 24 bits per pixel, mostrando quindi $2^{24} = 16M$
    di colori. 24 bit (16M di colori) è chiamato True Color.

    \subsection{Aliasing}
    L'introduzione di artefatti durante la ricostruzione di un segnale
    che, prima di venir campionato, conteneva oggetti più piccoli della metà
    del sampling rate (Nyquist). Nelle immagini la freq. è in relaz. alla dimensione
    strutturale. Parti piccole hanno freq. alta. Nelle immagini l'aliasing è spaziale.

    \subsection{Dither}
    Forma di rumore introdotta intenzionalmente per rendere casuale l'errore di
    quantizzazione.
    
    \subsection{Interpolazione}
    L’interpolazione è un'operazione spesso utilizzata per le immagini per eseguire 
    zoom, rotazioni, e altre correzioni geometriche. Viene eseguita utilizzando 
    informazioni conosciute in determinate posizioni per stimare i valori di 
    posizioni sconosciute.
    
    \f{interpolation}

    \subsection{Relazioni fra pixels}
    \textbf{Vicinanza} ogni pixel ha 2 vicini verticali, 2 vicini orizzontali e 4 vicini
    diagonali. \\
    \textbf{Adiacenza} due pixels sono adiacenti se sono vicini e rispettano determinati
    vincoli di prossimità di grigio o colore.

    \subsection{Distanza fra pixel}
    Euclidian Distance, City-Block distance, Chess-Board distance.
    
    \subsection{Formati file per immagini digitali}
    Raster: .gif, .jpg, .tiff, .png, .bmp

    \subsubsection{GIF}
    Crea una tabella fino a 256 colori, da un pool di 16M. Se l'immagine ha meno di 256
    colori l'algoritmo è capace di riprodurre l'immagine in maniera esatta.
    Per i casi in cui l'immagine contiene più di 256 colori, alcuni algoritm. 
    usano il colore più vicino mentre altri l'"error diffusion" per minimizzare
    l'errore. GIF sostituisce eventuali patterns (ampie zone di colore uniforme)
    con delle abbreviazioni (come $\text{bianco} \cdot 20$). Di conseguenza
    GIF è lossless solo per immagini con 256 colori o meno. Per immagini
    True Color può perdere il 99.998\% dei colori.

    \subsubsection{JPG}
    Formato ottimizzato per la fotografia ed immagini, contenenti colori simili
    in zone contigue.
    Lavora scartando informazione che l'occhio umano non dovrebbe notare.
    Salva l'informazione come colore a 24 bit. Compressione lossy, livello di compr.
    adattabile.

    \subsubsection{TIFF, PNG, BMP}
    \textit{TIFF} è un formato flessibile che può essere sia lossless che lossy.
    Viene principalmente usato senza nessun tipo di compressione.
    I dettagli dell'algoritmo utilizzato vengono salvati sul file.
    \\
    \textit{PNG} è lossless. Riduce la taglia del file cercando patterns che può sfruttare
    per il salvataggio dell'informazione.\\
    \textit{BMP} formato proprietario non compresso inventato da Microsoft. \\

    \subsection{Header e Container}
    Nelle immagini i file sono composti da una prima parte di meta informazione, 
    contenente le caratteristiche dell’informazione salvata, seguita dagli effettivi 
    dati dell’immagine.


    \subsection{Operazioni Array vs Matriciali}
    Le immagini raster possono essere viste come matrici. Su di esse è possibile
    eseguire operazioni di tipo array, sia matriciali. Ad es: il prodotto di due
    matrici può essere inteso come prodotto array (ogni elemento delle due matrici viene
    moltiplicato), oppure come l'effettivo prodotto matriciale.\\
    Di norma, nel digital signal processing si intendono le operazioni come
    operazioni di tipo array.

    \subsection{Operazioni lineari vs. Non lineari}
    Per un filtro lineare (ad esempio la media) si ha che:
    \begin{equation*}
      F_m (A + \lambda B) = F_m(A) + \lambda F_m(B)
    \end{equation*}
    Questa proprietà non viene invece soddisfatta dai filtri non lineari, come ad
    esempio la mediana.

    \subsection{Linear Image Processing}
    Le op. di linear image processing sono basate, come per il signal processing,
    su due op. fondamentali: la convoluzione e la trasformata di Fourier.\\
    Nell'image processing la conv. è l'op. più importante perché le immagini hanno
    l'informazione codificata nel dominio spaziale anziché quello delle frequenze.\\
    Filtri lineari possono ad es. venir utilizzati per:
    accentuare gli spigoli degli oggetti, ridurre il rumore di fondo.
    correggere l'illuminazione, ecc...
    Queste operazioni vengono eseguite per convoluzione tra l'immagine originale
    e un kernel di filtro, producendo l'immagine filtrata.

    \subsection{Analisi di Fourier}
    Meno utile per le immagini. Converte l'informazione chiara del dom. spaziale
    in una forma poco intellegibile nel dominio delle frequenze.

    \section{Tecniche base di image processing}

    \subsection{Digital Image Processing}
    \subsubsection{Equalizzazione dell'istogramma}
    È un metodo dell’image processing per l’adeguamento del contrasto. L'obiettivo
    è di ridistribuire le intensità nell'istogramma, fornendo, come conseguenza, 
    maggiore contrasto a zone che hanno basso contrasto locale. \\
    \textbf{Normalizzazione dell’immagine} 
    È un processo simile all’equalizzazione 
    dell’istogramma, in cui vengono modificate le intensità dei pixels (estendendone
    il range) per ottenere una migliore occupazione del range dinamico disponibile.
    \subsubsection{Blur}
    \textbf{Filtri digitali tramite convoluzione o DFT} 
    Per blur e sharpening si utilizzano tecniche di convoluzione o di DFT. Il 
    filtraggio può essere eseguito nel dominio spaziale tramite convoluzione, 
    applicando dei kernels, o nel dominio delle frequenze, tramite la 
    trasformata di Fourier (DFT). \\
    \textbf{Gaussian blur}
    Una sfocatura gaussiana si ottiene applicando all’immagine una funzione gaussiana 
    tramite convoluzione, facendo ciò si riduce il rumore ma anche la presenza di 
    dettagli nelle immagini. \\
    \textbf{Box blur}
    Un box blur è un filtro passa basso nel dominio spaziale, in cui ogni pixel
    è la media dei suoi vicini nell’immagine di partenza. Sono specificati tramite
    una matrice 3X3. \\
    \textbf{Sharpening}
    Un’immagine è a fuoco se ha buona risoluzione e un buon contrasto sui contorni 
    (edges). Lo sharpening permetti di migliorare/peggiorare questo contrasto. \\
    \textbf{Convoluzione}
    Nel trattamento d’immagine, la convoluzione è il processo di aggiungere ogni 
    elemento ai propri vicini, moltiplicato per i valori specificati in un kernel 
    di dimensione 3x3.
    \subsubsection{Kernel}
    I kernel usati nella convoluzione vengono anche detti matrici di convoluzione
    o maschere. Possono essere utilizzati per implementare una moltitudine di filtri,
    incluse alcune soluzioni per la detezione dei contorni.
    \f{kernels}
    \textbf{Normalizzazione del kernel} 
    La normalizzazione dei valori di un kernel è definita come la divisione di ogni 
    elemento del kernel per la somma di tutti gli elementi del kernel, in modo che 
    la somma di tutti i valori all’interno del kernel dia 1. Questa soluzione garantisce 
    che che i pixels siano in media luminosi nell’immagine risultante quanto lo sono 
    nell’immagine originale.
    \subsection{Detezione dei contorni}
    Il riconoscimento dei contorni (Edge detection) ha lo scopo di marcare i punti in 
    cui l'intensità luminosa cambia bruscamente. L'obbiettivo di questa tecnica è quella
    di rilevare i contorni quindi qualsiasi dettaglio non utile alla definizione di una
    forma o caratteristiche strutturali viene scartato.
    \subsection{DFT 2D}
    F è il risultato (lo spettro) della trasformata di Fourier dell'immagine f.
    L’operazione inversa permette di ricostruire l’immagine a partire dallo spettro.
    Il fatto che la modellazione eseguita dalla DFT sfrutti segnali periodici può 
    presentare delle complicazioni nel caso delle immagini, visto che non presentano 
    necessariamente periodicità. In maniera simile alla convoluzione, La DFT 2D 
    viene utilizzata per applicare varie tipologie di filtri.
    \subsection{Denoise}
    Immagini acquisite con camere digitali o convertite da film fotografico 
    convenzionale, contengono rumore introdotto da una moltitudine di fattori per es.
    quelli dati dai circuiti elettrici. Molto frequentemente il processing di queste 
    immagini richiede la rimozione (almeno parziale) di questo rumore, questo processo
    si chiama denoise.
    \subsection{Rumore salt-and-pepper}
    Il rumore di tipo salt-and-pepper, anche conosciuto come rumore impulsivo, causa 
    disturbi improvvisi e netti all’interno dell’immagine. Si presenta di norma come 
    occorrenze sparse di pixels neri o bianchi. Per risolverlo $\rightarrow$ filtro
    mediano.
    \subsection{Rumore gaussiano}
    Il rumore gaussiano ha una distribuzione statistica di tipo normale. 
    \subsection{Restoration: Wiener filter}
    I filtri più tradizionali eseguono la separazione tra segnale e rumore a condizione 
    che questi occupino diverse bande di frequenza. Il filtro di Wiener supera questa 
    limitazione con un approccio statistico. Si assume di avere conoscenza delle varie 
    caratteristiche e si cerca il filtro LTI il cui risultato sia "il più vicina 
    possibile" al segnale originale. Questo filtro è spesso utilizzato nel processo
    di deconvoluzione.
    \subsection{Segmentazione}
    La segmentazione è il processo in cui si suddivide un’immagine digitale in più 
    segmenti (sets di pixel). L'obiettivo è quello di semplificare/modificare la 
    rappresentazione dell’immagine in qualcosa che sia più facile e sensato 
    da analizzare.
    \subsection{Thresholding}
    Il metodo più semplice per eseguire una segmentazione è chiamato metodo di 
    sogliatura (Thresholding). Nella sua versione più essenziale, un’immagine 
    in scale di grigi viene convertita in un’immagine con pixels solo bianchi 
    o neri. Il valore usato come discriminante è la soglia (Threshold).
    \subsubsection{Basato sull’istogramma}
    I metodi di thresholding che sfruttano l’istogramma, sono normalmente efficaci
    rispetto agli altri metodi, perché necessitano di un solo passaggio di 
    processing sui pixels. Picchi e valli dell’istogramma vengono utilizzati 
    per identificare i clusters all’interno dell’immagine.
    \subsection{K-means}
    Abbiamo appena fatto Data Science quindi sappiamo tutti come funziona. Si usa
    per partizionare l'immagine in clusters.
    \subsubsection{Morphological matematica nell'Image Processing}
    La morfologia matematica è una teoria e tecnica per l’analisi e il processing 
    di strutture geometriche fondata sulla teoria degli insiemi, dei reticoli, 
    sulla topologia e sulle funzioni random.
    I quattro operatori principali sono: erosioni, dilatazione, apertura e chiusura.
    \subsubsection{Erosione}
    In uno spazio euclideo E, L’erosione dell’immagine A con lo structuring element 
    B è la combinazione di tutti i punti che può raggiungere il centro di B quando 
    si muove all’interno di A.
    \f{erosion}
    \subsubsection{Dilatazione}
    La dilatazione dell’immagine A con lo structuring element B è la combinazione 
    di tutti i punti che può raggiungere B quando il centro di B si muove all’interno 
    di A.
    \f{dilatation}
    \subsubsection{Apertura}
    L'apertura è la traslazione dello structuring element B all'interno di A.
    \f{open}
    \subsubsection{Chiusura}
    La chiusura è il complemento della traslazione dello structuring element B 
    all’esterno dall’immagine A.
    \f{closure}


    \section{Introduzione GPU e CUDA}

    \section{Programmare le GPU con CUDA}

    \section{Video Processing e Computer Vision}
    \subsection{Fenomeno Beta-Phi}
    Il cervello ricostruisce un'esperienza di movimento apparente se vede una 
    sequenza di immagini quasi identiche a 10/12 FPS.
    \subsection{Formati}
      \textbf{Proiettori}
      \begin{itemize}
        \item Per i film amatoriali: 8mm, super 8 e 9.5 mm
        \item Per il cinema: 35mm, 70mm
      \end{itemize}
      \textbf{Salvataggio}
      Betamax, cassette VHS, DVD e dischi Blu-ray ad alta definizione.
    \subsection{Tecnologie per il display}
    \begin{itemize}
      \item CRT (Cathode Ray Tube): tubo sottovuoto contenente uno o più emettitori 
      di fasci di elettroni accelerati o deflessi su uno schermo fluorescente.
      \item DLP (Digital Light Processing): tecnologia per video proiettori che 
      sfrutta micro-specchi digitali.
      \item PDP (Plasma Display Panel): pannello piatto che utilizza piccole celle 
      contenenti gas ionizzato caricato elettricamente.
      \item LCD (Liquid Crystal Display): pannello piatto che utilizza cristalli liquidi 
      modulati dalla luce generata dietro al pannello. 
      \item OLED (Organic Light-Emitting Diode): tecnologia che sfrutta la luce emessa 
      da LED, in combinazione con un elemento elettro-luminescente composto da uno 
      strato di materiale organico capace di produrre differenti tipi di luce in 
      risposta a una sollecitazione elettrica.
    \end{itemize}
    \subsection{Risoluzione}
    Rappresentata dalla quantità di pixel orizzontali. 
      \textbf{Nel cinema} 2K (2048×1080 quindi 2.2 megapixels) o 4K (4096×2160 quindi 
      8.8 megapixels). \\
      \textbf{Nelle televisioni e nei monitor} SD (Standard Definition): 576i, sistema 
      PAL/SECAM Europeo, 480i, sistema NTSC Americano, HD (High Definition): 720p - 
      1280×720p progressive scan quindi 921’600 pixels, 1080i - 1920×1080i interlacciato 
      quindi 1’036’800 pixels, 1080p - 1920×1080p quindi 2’073’600 pixels, UHD (Ultra 
      High Definition): 4K - 3840×2160 progressive scan, True 4K - 4096×2160, 8K - 
      7680×4320, True 8K - 8192×4320.
    \subsection{Aspect ratio}
    Descrive la proporzione fra la larghezza e l’altezza di un’immagine.
    \begin{itemize}
      \item 4:3 è il formato utilizzato dall’invenzione della cinematografia, i vecchi 
      film a 35 mm.
      \item 16:9 è il formato standard internazionale dell’HDTV e del DVD.
      \item 21:9 è un formato cinematografico recente utilizzato oggi negli 
      ultra-wide monitors.
    \end{itemize}
    
    \subsection{Framerate}
    Frequenza con cui le immagini vengono mostrate in un filmato. 

    \subsection{Video processing}  come l’image processing si utilizzano tecniche di 
    filtraggio dei segnali applicandole a streams di tipo video. Le tecniche di video 
    processing vengono impiegate per: modifica dell’aspect ratio, zoom digitale, 
    aggiustamento di brightness, contrast, hue, saturation, sharpness, gamma, 
    conversione del frame rate, aggiustamento del colore, riduzione del rumore, 
    tecniche di upscaling. \\
    Le tecniche attuate a livello di singola immagine vengono dette tecniche di 
    \textit{intra-frame processing}. Mentre quelle che sfruttano l’informazione 
    temporale esistente fra più immagini in sequenza vengono dette tecniche di 
    \textit{inter-frame processing}.

    \subsubsection{Formati di codifica e compressione video}
    Modo di rappresentare i dati per il salvataggio e la trasmissione. Sono 
    definiti da una specifica e il programma che li codifica prende il nome di 
    codec (MPEG-2 Part 2, MPEG-4 Part 2, H.264 (MPEG-4 Part 10), HEVC e RealVideo RV40).

    \subsubsection{Formati container per il multimedia}
    Contengono il video combinato con uno stream audio (AVI, MP4, FLV, RealMedia, 
    Matroska, o QuickTime).

    \subsubsection{Compressione video}
    Esistono sia compressori lossy che lossless. Alcuni formati video come il 
    Dirac e l’H264 permettono entrambe le tipologie.

    \subsubsection{Formati intra-frame}
    La compressione viene applicata esclusivamente alle singole immagini senza 
    approfittare della correlazione tra immagini successive. Questa soluzione è meno 
    computationally intensive delle soluzioni che applicano interframe prediction.

    \subsubsection{Inter-frame prediction}
    I video contengono molta informazione ridondante spaziale, ma soprattutto temporale. 
    Algoritmi di compressione come H.264 sfruttano questa ridondanza con tecniche 
    di inter-frame prediction che gli permettono di raggiungere rapporti di compressione 
    di 1:50.

    \subsubsection{Motion JPEG}
    Formato di compressione video intraframe in cui ogni frame è compresso come 
    immagine JPEG ha un’efficienza limitata a 1:20. 

    \subsubsection{Computer Vision}
    Tecnologie e metodi per acquisire, modificare ed analizzare immagini con lo 
    scopo di estrapolare informazioni per poter prendere decisioni in maniera automatica.

    \subsubsection{Struttura di un sistema di computer vision}
    \begin{enumerate}
      \item Acquisizione dell’immagine
      \item Pre-processing: rimozione del rumore, l’adeguamento del contrasto, ecc.
      \item Estrazione delle feature: estrazione di caratteristiche (linee, contorni).
      \item Detection/Segmentation: decidere quali punti o regioni dell’immagine sono 
      rilevanti.
      \item Processing di tipo high-level: classificare gli oggetti, determinarne 
      posizione, dimensione …
    \end{enumerate}
    
    \subsubsection{Background substraction}
    Estrae gli elementi in primo piano all’interno di un video confrontando il frame 
    corrente con un frame di riferimento (background). Tecniche di background subtraction: 
    \begin{itemize}
      \item Frame differencing (sottrazione tra pixel dell’immagine attuale e quella di 
      background).
      \item Mean filter (tecnica di analisi non parametrica per identificare i maxima 
      (detti modes) di una funzione di densità).
      \item Gaussian average.
    \end{itemize}

    \subsubsection{Video motion Traking}
    Tecnica di video tracking che associa gli oggetti in frames video consecutivi. I 
    componenti principali sono: rappresentazione e localizzazione del target, 
    filtraggio e associazione dei dati.
    \section{Audio e Video Streaming}

    \section{Gstreamer}
    \begin{itemize}
     \item Framework open source per la creazione di applicazioni multimediali
     \item Piattaforma altamente modulare sviluppata in C
     \item Multi-platform: Linux-kernel based, Android, MacOS, iOS, Windows
     \item Esistono dei bindings per vari linguaggi: ad esempio Python, C++, Perl, Ruby, C#, …
     \item Utilizzabile anche da linea di comando
     \item Utilizza un’architettura a plug-ins
     \item Molti elementi quali codecs, container formats, input e output drivers ed effetti sono forniti al momento dell’installazione
     \item Possibilità di sviluppare ulteriori plug-ins utilizzando il linguaggio C
    \end{itemize}
    \textbf{Elemento} è il componente base di una pipeline, ha delle proprietà alle queali si possono assegnare dei valori.
    3 elementi di base: source, filter e sink \\
    \textbf{Pipelines} insiemi di elementi collegati sequenzialmente. \\
    Pipeline di base -> almeno una source, almeno un sink e uno o più elementi filter 
    descritta da una serie di caratteri separati da !
    \subsection{Elementi (audio)}
    Generazione segnali audio di test:\\
    \textbf{audiotestsrc} -> generatore di suoni (utilizzato per testare delle pipelines audio).
    Il tipo di suono è scelto utilizzando la proprietà wave\\
    
    Audio adapters:\\
    \textbf{audioconvert} -> converte dei dati audio raw in vari formati\\
    \textbf{audioresample} -> esegue una conversione di sample rate\\
    Ascolto di audio:\\
    \textbf{autoaudiosink} -> seleziona automaticamente la device di riproduzione dell’audio

    \subsection{Elementi (video)}
    Generazione video di test:\\
    \textbf{videotestsrc} -> generatore di pattern video (utilizzato per testare delle pipelines
    video). Il tipo di pattern è scelto utilizzando la proprietà pattern\\
    
    Video adapters:\\
    \textbf{videoconvert} -> converte da un color space ad un altro (ad esempio da RGB a YUV)\\
    \textbf{videorate} -> esegue una conversione di sample rate (eliminando o duplicando dei frames)\\
    \textbf{videoscale} -> ridimensiona il video\\

    Visualizzazione dei media:
    \textbf{autovideosink} -> seleziona automaticamente la device di riproduzione del video

    \subsection{Elementi utili}

    \textbf{Bins}: elementi in grado di costruire dinamicamente una pipeline in base alle informazioni estratte dal media\\
    
    \textbf{playbin} -> gestisce tutti gli aspetti di riproduzione di un media (dalla lettura, alla decodifica fino alla riproduzione)\\
    \textbf{uridecodebin} -> decodifica i dati ricevuti da un URI fino ad ottenere dei dati raw\\
    \textbf{decodebin} -> decodifica i dati in entrata fino ad ottenere dei dati raw
    
    \subsection{File input / output}
    
    \textbf{filesrc} -> legge i dati presenti in un file\\
    \textbf{filesink} -> scrive i dati in input su di un file\\

    \subsection{Gst tools} 

    \textbf{gst-inspect-1.0} -> Mostra una lista di tutti gli elementi disponibili\\
    
    \textbf{gst-inspect-1.0 [PLUGIN | ELEMENT]} -> Mostra le info disponibili per il plugin/elemento selezionato.\\
    
    \textbf{gst-launch-1.0 PIPELINE_DESCRIPTION} -> Tool per creare ed eseguire una pipeline
    Utilizzando l’opzione -v si attiva il supporto verbose\\
    
    \subsection{Debug} 
    
    \textbf{GST_DEBUG} -> Variabile d’ambiente utilizzata per specificare il livello di debug desiderato\\
    
    Level 	Name 	Description\\
    0       none	No debug information is output.\\
    1       ERROR Logs all fatal errors.\\
    2       WARNING Logs all warnings.\\
    4       INFO 	Logs all informational messages.\\
    5       DEBUG Logs all debug messages.\\
    6       LOG   Logs all log messages.\\
    
    \textbf{GST_DEBUG_DUMP_DOT_DIR} ->  Variabile d’ambiente utilizzata per specificare la path dove salvare il grafico della pipeline
    

    \section{Machine Learning}
    Campo dell'informatica con l'obiettivo di sviluppare funzionalità per rendere
    i computers capaci di imparare senza che li si programmi in maniera esplicita.
    La realizzazione di buoni algoritmi di machine learning è però un compito complesso
    perché riconoscere i patterns può essere difficile e spesso i dati disponibili
    non sono sufficienti.
    Tipi di machine learning:
    \begin{itemize}
    \item Supervised learning
    \item Reinforcement learning
    \item Unsupervised learning
    \end{itemize}
    
    \subsection{Supervised learning}
    Nel supervised learning all’algoritmo vengono forniti sia esempi di input, che i
    rispettivi dati di output, da un “educatore” dell’algoritmo. L’algoritmo cerca di
    scoprire la regola generale che associa gli inputs agli outputs.
    \f{SupervisedLearning}
    
    \subsection{Reinforcement learning}
    Nelle forme più estreme, come ad esempio nel reinforcement learning, come unica
    forma di feedback l’output viene approvato o respinto. Il reinforcement learning
    viene ad esempio sfruttato nei sistemi di guida automatica.
    \f{ReinforcementLearning}
    
    \subsection{Unsupervised learning}
    Nel unsupervised learning, nessuna informazione viene fornita all’algoritmo, che
    è costretto a scoprire in maniera autonoma la struttura dei dati in input. È la
    forma più complessa di machine learning.
    \f{UnsupervisedLearning}
    
    \subsection{Reti artificiali}
    Una delle famiglie più utilizzate di algoritmi di machine learning è quella delle
    reti neurali. I calcoli vengono strutturati in termini di gruppi di neuroni artificiali
    interconnessi, che processano collettivamente l’informazione. Le neural networks
    utilizzate oggi sono strumenti non lineari per la modellazione di tipo statistico
    di dati con una certa complessità.\\
    Vantaggi: generalmente più efficaci e semplici da allenare degli altri algoritmi
    di machine learning. Offrono più flessibilità di parametrizzazione.\\
    Svantaggi: la loro natura di tipo black-box può risultare svantaggiosa in casi in cui
    è necessario il controllo del comportamento dell’algoritmo (predictability). Per di
    più le reti neurali artificiali sono facilmente soggette ad overfitting. Con pochi dati
    non identificano la caratteristica generica desiderata ma il caso specifico.
    Per renderle capaci di generalizzare è necessario allenarle con data sets molto ampi e ben costruiti.
    
    \subsection{Deep Learning}
    La crescita delle capacità di calcolo, la progressiva riduzione del costo dell’hardware
    e l’evoluzione delle GPUs degli ultimi anni, hanno contribuito allo sviluppo del concetto
    di deep learning. Vengono utilizzate reti neurali composte da layers multipli. Ogni layer
    impara a trasformare i dati di input in una rappresentazione ogni volta leggermente più
    astratta e composta. Nelle soluzioni end-to-end l’intero processo, dai sensori agli attuatori,
    viene eseguito da un’unica layered o recurrent neural network senza modularizzazione.
    Vantaggio: capacità di continuare a migliorare le performances all’aumentare dei dati di
    apprendimento disponibili.\\
    Per quanto concerne l’ambito del multimedia, oggi esistono applicazioni di successo soprattutto
    nella computer vision e nel riconoscimento vocale. Ad esempio, il deep learning viene utilizzato
    con successo nella classificazione di immagini.\\
    Esempi:
    \begin{itemize}
    \item Amazon Rekognition: aggiunge funzionalità di analisi d’immagine alle applicazioni.
    \item Amazon Transcribe: analizza file audio e li traduce in file di testo.
    \item Caffe: framework di deep learning veloce, modulare e facile da utilizzare.
    \end{itemize}
\end{multicols}
\end{document}
